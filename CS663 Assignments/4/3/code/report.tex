\documentclass[10pt]{report}
\usepackage{amsmath}

\begin{document}
\title{CS663 Assignment-4 Question-3}
\author{KOTWAL ALANKAR SHASHIKANT}
\maketitle

\section*{Part A}
Consider a matrix \textbf{A} of size $m \times n$. 	Define $\textbf{P}=\textbf{A}^T \textbf{A}$ and $\textbf{Q}=\textbf{A} \textbf{A}^T$.

Now, if \textbf{A} has dimensions $m \times n$, \textbf{P} has dimensions $n \times n$. Hence if we want to evaluate $\textbf{y}^T \textbf{P} \textbf{y}$ for a (column) vector \textbf{y}, \textbf{y} must have dimensions $n \times 1$. This means the product $\textbf{y}^T \textbf{P} \textbf{y}$ will be a scalar.

Now we have $$\textbf{y}^T \textbf{P} \textbf{y} = \textbf{y}^T \textbf{A}^T \textbf{A} \textbf{y}$$
Putting $\textbf{Z} = \textbf{A}\textbf{y}$, and noticing that $\textbf{Z}^T = \textbf{y}^T \textbf{A}^T$, we have $$\textbf{y}^T \textbf{P} \textbf{y} = \textbf{Z}^T \textbf{Z}$$

Thus the given expression is the dot product of a vector with itself, which is always non-negative. Thus, $$\textbf{y}^T \textbf{P} \textbf{y} \geq 0$$

Similarly, we have for a $n \times 1$ vector \textbf{z}, with $\textbf{Y} = \textbf{A}\textbf{z}$ $$\textbf{z}^T \textbf{Q} \textbf{z} = \textbf{Y}^T \textbf{Y}$$

which is again the dot product of a vector with itself, which is always non-negative. Thus, $$\textbf{z}^T \textbf{Q} \textbf{z} \geq 0$$

Now consider an eigenvector $\textbf{z}$ of $\textbf{A}^T \textbf{A}$, where \textbf{z} is a $m \times 1$ vector. We must have for some $\lambda$
$$\textbf{A}^T \textbf{A} \textbf{z} = \lambda \textbf{z}$$

A pre-multiplication by $\textbf{z}^T$ yields
$$\textbf{z}^T \textbf{A}^T \textbf{A} \textbf{z} = \textbf{z}^T \lambda \textbf{z}$$

which means
$$\textbf{z}^T \textbf{A}^T \textbf{A} \textbf{z} = \lambda \textbf{z}^T \textbf{z}$$

Now the left hand side is positive, as proved above. $\textbf{z}^T \textbf{z}$ is positive, because it is the dot product of a vector with itself. Hence, $\lambda$ must be non-negative.

Similarly for $\textbf{A} \textbf{A}^T$, consider an eigenvector $\textbf{z}$ of dimension $m \times 1$. Multiplying the eigenvalue equation on the left by $\textbf{z}^T$ we get for some $\lambda$
$$\textbf{z}^T \textbf{A} \textbf{A}^T \textbf{z} = \lambda \textbf{z}^T \textbf{z}$$
which again means $\lambda$ is non-negative.

Hence the eigenvalues of $\textbf{A}^T \textbf{A}$ and $\textbf{A} \textbf{A}^T$ are non-negative.

\newpage
\section*{Part B}
If \textbf{u} is an eigenvector of \textbf{P} with eigenvalue $\lambda$, we have
$$\textbf{A}^T \textbf{A} \textbf{u} = \lambda \textbf{u}$$
Pre-multiply by $\textbf{A}^T$ to give
$$\textbf{A} \textbf{A}^T \textbf{A} \textbf{u} = \textbf{A} \lambda \textbf{u}$$
implying
$$\textbf{A} \textbf{A}^T \textbf{A} \textbf{u} = \lambda \textbf{A} \textbf{u}$$
Thus $\textbf{Au}$ is an eigenvector of \textbf{Q} with eigenvalue $\lambda$.

Similarly, if \textbf{v} is an eigenvector of \textbf{Q} with eigenvalue $\mu$, we have by pre-multiplying with $\textbf{A}^T$
$$\textbf{A}^T \textbf{A} \textbf{A}^T \textbf{v} = \mu \textbf{A}^T \textbf{v}$$
Thus $\textbf{A}^T \textbf{v}$ is an eigenvector of \textbf{P} with eigenvalue $\mu$.
\linebreak
Clearly \textbf{u} has $n$ elements and \textbf{v} has $m$ elements.

\section*{Part C}
If $\textbf{v}_i$ is an eigenvector of \textbf{Q}, we have
$$\textbf{AA}^T \textbf{v}_i = \lambda_i \textbf{v}_i$$
Defining $\textbf{u}_i$ as $$\textbf{u}_i = \frac{\textbf{A}^T \textbf{v}_i}{||\textbf{A}^T \textbf{v}_i||}$$ and substituting it in the previous equation yields
$$\textbf{Au}_i = \frac{\lambda_i}{||\textbf{A}^T \textbf{v}_i||} \textbf{v}_i$$
Define $\gamma_i$ as $$\gamma_i = \frac{\lambda_i}{||\textbf{A}^T \textbf{v}_i||}$$
Clearly $\gamma_i$ are non-negative because $\lambda_i$ are non-negative and the denominator being a norm is non-negative as well.
Thus there exist some real, non-negative $\gamma_i$ which satisfy
$$\textbf{Au}_i = \gamma_i \textbf{u}_i$$
\newpage

\section*{Part D}
Consider the product $\textbf{U} \Gamma \textbf{V}^T$ with the matrices as defined in the question. We have
$$\textbf{U} \Gamma = [ \textbf{v}_1 | \textbf{v}_2 | \cdots | \textbf{v}_m ] \begin{pmatrix}
\gamma_{1} & 0 & \cdots & 0 \\
0 & \gamma_{2} & \cdots & 0 \\
\vdots  & \vdots  & \ddots & \vdots  \\
0 & 0 & \cdots & \gamma_{m}
\end{pmatrix}$$
Thus
$$\textbf{U} \Gamma = [ \gamma_1 \textbf{v}_1 | \gamma_2 \textbf{v}_2 | \cdots | \gamma_m \textbf{v}_m ] $$
From the previous part each $\gamma_{i} \textbf{v}_i$ can be written as
$$\gamma_{i} \textbf{v}_i = \textbf{Au}_i$$
Thus $$\textbf{U} \Gamma = [ \textbf{Au}_1 | \textbf{Au}_2 | \cdots | \textbf{Au}_n ] $$
which means
$$\textbf{U} \Gamma \textbf{V}^T = [ \textbf{Au}_1 | \textbf{Au}_2 | \cdots | \textbf{Au}_n ]  [ \textbf{u}_1 | \textbf{u}_2 | \cdots | \textbf{u}_n ]^T$$
implying
$$\textbf{U} \Gamma \textbf{V}^T = \textbf{A} [ \textbf{u}_1 | \textbf{u}_2 | \cdots | \textbf{u}_n ]  [ \textbf{u}_1 | \textbf{u}_2 | \cdots | \textbf{u}_n ]^T$$
Now given that $$\textbf{u}_i \textbf{u}_j = \delta_{ij}$$ we have

$$A = \textbf{U} \Gamma \textbf{V}^T$$
where $\Gamma$ is a matrix that has the non-negative $\gamma_i$ on its diagonal elements and zeros on the off-diagonal elements.
Thus we are done.

\end{document}