\documentclass[10pt]{report}
\usepackage{amsmath}

\begin{document}
\title{CS663 Assignment-4 Question-2}
\author{KOTWAL ALANKAR SHASHIKANT}
\maketitle

We have the following relations:
$$g_1 = f_1 + f_2 * h_2$$
$$g_2 = f_1 * h_1 + f_2$$
In the Fourier domain this means
$$G_1 = F_1 + F_2 H_2$$
$$G_2 = F_1 H_1 + F_2$$
Solving these equations simultaneously for $F_1$ and $F_2$ yields:
$$F_1 = \frac{H_2 G_2 - G_1}{H_1 H_2 - 1}, F_2 = \frac{H_1 G_1 - G_2}{H_1 H_2 - 1}$$
and $$f_1 = \mathcal{F}^{-1}(F_1), f_2 = \mathcal{F}^{-1}(F_2)$$
These inverse filters will blow up wherever $H_1H_1$ is close to 1. This is one problem with this technique for reflection removal.

Now the problem here is that the actual images we have will be
$$g_1 = f_1 + f_2 * h_2 + n_1$$
$$g_2 = f_1 * h_1 + f_2 + n_2$$
where the added terms are additive noise. Each image has a different, and independent, noise image and hence it is impossible to estimate both $n_1$ and $n_2$ given just these two images.
\end{document}